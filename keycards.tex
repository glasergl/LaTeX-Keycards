%
% this command evalutes to the value of the section label, depending on the section counters
% intermediate counters are needed, because \thesubsection and \thesubsubsection dont evalute to just the number of the corresponding counter
%
\newcounter{subsectionn}
\newcounter{subsubsectionn}
\newcommand{\sectionlabel}{%
\setcounter{subsectionn}{\value{subsection}}%
\setcounter{subsubsectionn}{\value{subsubsection}}%
\ifnum\value{section}=0%
{}%
\else%
\thesection%
\fi%
\ifnum\value{subsectionn}=0%
{}%
\else%
{.\thesubsectionn}%
\fi%
\ifnum\value{subsubsectionn}=0%
{}%
\else%
{.\thesubsubsectionn}%
\fi%
}

%
% this command evaluates to the title, given the title name, besides, the title is rotated depending on the counter flag
% #1 - title
%
\newcommand{\keycardtitle}[1]{%
	\ifnum\value{printTitleRotated}=1%
		\hspace*{\fill}\rotatebox[origin=c]{180}{\textbf{\sectionlabel\ #1}}%
	\else%
		\textbf{\sectionlabel\ #1}%
	\fi%
}

%
% #1 - title of the key card, i.e., the reference of the keycard's content
% #2 - content of the key card
%
\newcommand{\keycard}[2]{%
\parbox[t][0.248\textheight][t]{0.48\textwidth}{%
	\keycardtitle{#1}
	
	\vspace{0.5cm}%
	#2% 
}%
}

%
% Behaves similar as \keycard with the difference that the given content may span the whole text widht and also multiple pages
%
\newcommand{\longkeycard}[2]{%
\vspace{0.5cm}%
\hrule%
\vspace{0.5cm}%
\keycardtitle{#1}%
		
\vspace{0.5cm}%
#2%
\vspace{0.5cm}%
\hrule%
\vspace{0.5cm}%
}